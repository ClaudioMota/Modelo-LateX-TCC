\chapter{Conhecendo o LateX}\label{capConhecendoLatex}
Neste capítulo você irá conhecer coisas de outro mundo. Você irá escrever um livro como se estivesse programando. Eu sei! Você deve estar pensando: "Que loco cara!" mas é isso mesmo!!!

Nada mais de ter que se preocupar com formatação, como organizar capítulos, referências cruzadas, entre outras maravilhas que tiram o socego de quem detesta escrever artigos. Mas nem tudo é alegria\ldots Neste documento você também irá sentir saudade do corretor ortográfico do Word com aquele toque de correção semantica\ldots Bons tempos quando você tinha alguém que te dizia que você estava escrevendo errado. Só pra citar~\cite{camargo}.

Por isso lembre-se! Se você escrever algo do tipo: "\underline{Chamei a secretaria para me ajudar, ela é uma pessoa muito colaborativa}". O LateX não irá te avisar que existe um erro semântico alí. E dependendo do editor que você estiver usando não irá nem indicar um ero ortográfico. Por isso fique atento(a)!

Ah e antes que me esqueça... Não podemos esquecer de uma tradição tão antiga:

\textbf{Olá mundo!}

É importante que conforme você avance na leitura deste documento, também fique atento no que está ocorrendo no \textbf{\textit{código fonte}} do mesmo. Pois iremos apresentar as novas funcionalidades a partir de exeplos.

