%%%%%%%%%%%%%%%%%%%%%%%%%%%%%%%%%%%%%%%%%%%%%%%%%%%%%%%%%
%% Modelo DCOMP/TCC da Universidade Federal de Sergipe %%
%%%%%%%%%1%%%%%%%%%%%%%%%%%%%%%%%%%%%%%%%%%%%%%%%%%%%%%%%
%-------------------------------------------------------------------
% Baseado no modelo do DCOMP/PROCC da Universidade Federal de Sergipe
% Ultima modificação: 13/04/2016
% Colaboradores:
%   <Desenvolvedores do modelo DCOMP/PROCC>
%   Claudio Mota Oliveira (cmoliveira1000@hotmail.com)
%-------------------------------------------------------------------

\documentclass[a4paper,titlepage]{tcc_dcomp}
\usepackage[brazil,english]{babel}
\usepackage{procc,epsfig}
\usepackage{times}
\usepackage{verbatim}
\usepackage{mathtools}
\usepackage{amsmath,amsfonts,amsthm}

\newcommand{\etal}{et~al.}

%-------------------------- Para usar acentuacaoo em sistemas ISO8859-1 ------------------------------------
% Se estiver usando o Microsoft Windows ou linux com essa codificacao, descomente essa linhas abaixo
% e comente as linhas referentes ao UTF8
%\usepackage[latin1]{inputenc} 
% Usar acentuacao em sistemas ISO8859-1, comentar a linha com  \usepackage[utf8x]{inputenc}
%-----------------------------------------------------------------------------------------------------

%-------------------------- Para usar acentuacao em sistemas UTF8 ------------------------------------
% Para a maior parte das distribuicoes linux, usar a opcao utf8x (lembrar de comentar as linha referente a ISO8859-1 acima)
\usepackage{ucs}
\usepackage[utf8x]{inputenc}
%\usepackage[utf8]{inputenc}
\usepackage[T1]{fontenc}
%-----------------------------------------------------------------------------------------------------

% usado nas tabelas
\usepackage{multirow}
\usepackage{float}
\usepackage{graphicx,url}
\usepackage{subfig}
\graphicspath{{Figuras/}}
\usepackage{longtable} %tabelas longas, para tabelas que ultrapassam uma pagina
%\input{psfig.sty}


\usepackage{amssymb} %% Usando nas formulas matematicas


% ----------------- Para inserir codigo fonte de linguagens de programacao no documento -------------
\usepackage{listings}
\lstset{numbers=left,
stepnumber=1
firstnumber=1,
%numberstyle=\tiny,
extendedchars=true,
breaklines=true,
frame=tb,
basicstyle=\footnotesize,
stringstyle=\ttfamily,
showstringspaces=false
}
\renewcommand{\lstlistingname}{C\'odigo Fonte}
\renewcommand{\lstlistlistingname}{Lista de C\'odigos Fonte}
% ---------------------------------------------------------------------------------------------------

%\selectlanguage{portuges}
\sloppy

\begin{document}

%%%%%%%%%%%%%%%%%%%%%%%%%%%%%%%%%%%%%%%%%%%%%%%%%%%%%%%%%%%%%%%%%%%%%%%%%%%%%%%%
%% Informacoes para a Pagina de Rosto
%%
\Titulo{INSIRA O TÍTULO DO SEU TCC AQUI}
\Autor{INSIRA SEU NOME EM MAIÚSCULO AQUI}
\Data{\today} % Data da Defesa por extenso
\Ano{2016}
\Orientador{Prof.(ª) Dr.(ª) Nome Orientador(a)} %Ex: Prof. Dr. João Farias
\UniverOrientador{Universidade Federal de Sergipe (UFS)} %Universidade do orientador

%Se possuir um professor co-orientador descomente essas duas linhas
%\CoOrientador{Prof. Dr. Nome Co-orientador}
%\UniverCoOrientador{Universidade Fictícia de Orientadores Brasileira (UFOB)}

%Não utilizado
%\priAvaliador{Prof. Dr. Nome do primeiro Avaliador}
%\UniverpriAvaliador{Universidade Federal de Sergipe (UFS)}
%\segAvaliador{Prof. Dr. Nome do segundo Avaliador}
%\UniversegAvaliador{Universidade Federal de Sergipe (UFS)}

\newpage
\cleardoublepage

\PaginadeRosto

\newpage
\cleardoublepage

%[~debug] Verificar se existem esta parte do documento, salvo engano existe
%%%%%%%%%%%%%%%%%%%%%%%%%%%%%%%%%%%%%%%%%%%%%%%%%%%%%%%%%%%%%%%%%%%%%%%%%%%%%%%%%%%%%%%%
%% Dedicatoria
%% Na dissertação retire o simbolo % do bloco abaixo e crie o arquivo dedicatoria.tex 
%%
%\begin{dedicatoria}
%\input{dedicatoria.tex}
%\end{dedicatoria}

\newpage
\cleardoublepage

%[~debug] Verificar se existem esta parte do documento, salvo engano existe
%%%%%%%%%%%%%%%%%%%%%%%%%%%%%%%%%%%%%%%%%%%%%%%%%%%%%%%%%%%%%%%%%%%%%%%%%%%%%%%%%%%%%%%%
%% Agradecimentos
%% Na dissertação retire o simbolo % do bloco abaixo e crie o arquivo agradecimentos.tex 
%%
%\begin{agradecimentos}
%\input{agradecimentos.tex}
%\end{agradecimentos}

\newpage
\cleardoublepage

%%%%%%%%%%%%%%%%%%%%%%%%%%%%%%%%%%%%%%%%%%%%%%%%%%%%%%%%%%%%%%%%%%%%%%%%%%%%%%%%%%%%%%%%
%% Resumo
%% Na dissertação retire o simbolo % do bloco abaixo e crie o arquivo resumo.tex 
%%
\begin{resumo} 
Este projeto tem como objetivo facilitar a vida dos futuros formandos do departamento de computação da Universidade Federal de Sergipe.
Nos próximos capítulos você deverá aprender os conceitos básicos do LateX e como usar suas principais ferramentas.
Esperamos que seja proveitoso, e que você não mais precise usar o Word ou o LibreOffice Writer!! :-)
É aconselhável que, no resumo, seja escrita uma frase por linha!


\end{resumo}

\newpage
\cleardoublepage

%%%%%%%%%%%%%%%%%%%%%%%%%%%%%%%%%%%%%%%%%%%%%%%%%%%%%%%%%%%%%%%%%%%%%%%%%%%%%%%%%%%%%%%%
%% Abstract
%% Na dissertação retire o simbolo % do bloco abaixo e crie o arquivo abstract.tex 
%%
\begin{abstract}
This project was made in order to make your life easier. 
Stop reading this and go learn something useful on the next chapters. 
Enjoy it! :-)
\end{abstract}

\newpage
\cleardoublepage

%%%%%%%%%%%%%%%%%%%%%%%%%%%%%%%%%%%%%%%%%%%%%%%%%%%%%%%%%%%%%%%%%%%%%%%%%%%%%%%%%%%%%%%
%% Lista de Figuras, Lista de Tabelas, Lista de Siglas e Sumario
%%
\selectlanguage{brazil}

\listoffigures
\listoftables
\ListadeSiglas
%\lstlistoflistings %lista de codigos fonte - Para inserir a listagem de codigos fonte
\renewcommand{\contentsname}{Sumário}
\Sumario

\Introducao

\newpage
\cleardoublepage

%%%%%%%%%%%%%%%%%%%%%%%%%%%%%%%%%%%%%%%%%%%%%%%%%%%%%%%%%%%%%%%%%%%%%%%%%%%%%%%%%%%%%%%%%
%% Definicao do cabecalho: secao do lado esquerdo e numero da pagina do lado direito
\pagestyle{fancy}
\addtolength{\headwidth}{\marginparsep}\addtolength{\headwidth}{\marginparwidth}\headwidth = \textwidth
\renewcommand{\chaptermark}[1]{\markboth{#1}{}}
\renewcommand{\sectionmark}[1]{\markright{\thesection\ #1}}\lhead[\fancyplain{}{\bfseries\thepage}]%
	     {\fancyplain{}{\emph{\rightmark}}}\rhead[\fancyplain{}{\bfseries\leftmark}]%
             {\fancyplain{}{\bfseries\thepage}}\cfoot{}

%%%%%%%%%%%%%%%%%%%%%%%%%%%%%%%%%%%%%%%%%%%%%%%%%%%%%%%%%%%%%%%%%%%%%%%%%%%%%%%%%%%%%%%%%
%
% Hifenizacao - Colocar lista de palavras que nao devem ser separadas e que 
% nao estao no dicionario portugues.
% As palavras do dicionario portugues ja sao separadas corretamente pelo lateX
%
\hyphenation{ Hardware Software }

%%%%%%%%%%%%%%%%%%%%%%%%%%%%%%%%%%%%%%%%%%%%%%%%%%%%%%%%%%%%%%%%%%%%%%%%%%%%%%%%%%%%%%%%%%
%Corpo do TCC:
\chapter{Conhecendo o LateX} \label{capConhecendoLatex}
Neste capítulo você irá conhecer coisas de outro mundo. Você irá escrever um livro como se estivesse programando. Eu sei! Você deve estar pensando: "Que loco cara!" mas é isso mesmo!!!

Nada mais de ter que se preocupar com formatação, como organizar capítulos, referências cruzadas, entre outras maravilhas que tiram o socego de quem detesta escrever artigos. Mas nem tudo é alegria... Neste documento você também irá sentir saudade do corretor ortográfico do Word com aquele toque de correção semantica... Bons tempos quando você tinha alguém que te dizia que você estava escrevendo errado.

Por isso lembre-se! Se você escrever algo do tipo: "\underline{Chamei a secretaria para me ajudar, ela é uma pessoa muito colaborativa}". O LateX não irá te avisar que existe um erro semântico alí. E dependendo do editor que você estiver usando não irá nem indicar um ero ortográfico. Por isso fique atento(a)!

Ah e antes que me esqueça... Não podemos esquecer de uma tradição tão antiga:

\textbf{Olá mundo!}

É importante que conforme você avance na leitura deste documento, também fique atento no que está ocorrendo no \textbf{\textit{código fonte}} do mesmo. Pois iremos apresentar as novas funcionalidades a partir de exeplos.



%%%%%%%%%%%%%%%%%%%%%%%%%%%%%%%%%%%%%%%%%%%%%%%%%%%%%%%%%%%%%%%%%%%%%%%%%%%%%%%%%%%%%%%%%
%% Troca do termo Bibliografia para Referencia.
%% Adicao da pagina da Referencia no Sumario
\addcontentsline{toc}{chapter}{Referências}
\renewcommand{\bibname}{REFERÊNCIAS} 

%%%%%%%%%%%%%%%%%%%%%%%%%%%%%%%%%%%%%%%%%%%%%%%%%%%%%%%%%%%%%%%%%%%%%%%%%%%%%%%%%%%%%%%%%%%%%%%%
%% Bibliografia
%% Coloque suas referencias no arquivo referencias.bib
\bibliographystyle{plain} % estilo de bibliografia   plain,unsrt,alpha,abbrv.
%\nocite{*} %coloca todas as referências
\bibliography{Documentos/Referencias} % arquivos com as entradas bib.

%%%%%%%%%%%%%%%%%%%%%%%%%%%%%%%%%%%%%%%%%%%%%%%%%%%%%%%%%%%%%%%%%%%%%%%%%%%%%%%%%%%%%%%%%%%%%%%%%
%% Apendice
%% Caso seja necessario algum apendice, descomente a linha abaixo e crie o arquivo apendices.tex.
\appendix
%\input{apendices}

\end{document}