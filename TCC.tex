%%%%%%%%%%%%%%%%%%%%%%%%%%%%%%%%%%%%%%%%%%%%%%%%%%%%%%%%%
%% Modelo DCOMP/TCC da Universidade Federal de Sergipe %%
%%%%%%%%%1%%%%%%%%%%%%%%%%%%%%%%%%%%%%%%%%%%%%%%%%%%%%%%%

% ----- ----- ----- ----- Informacoes importantes ----- ----- ----- ----- %
% Baseado no modelo do DCOMP/PROCC da Universidade Federal de Sergipe
% Ultima modificacao: 24/04/2016
% Colaboradores:
%   <Desenvolvedores do modelo DCOMP/PROCC>
%   Claudio Mota Oliveira (cmoliveira1000@hotmail.com)
%	Fernando Melo Nascimento (nascimentofm@ufs.br)
% ----- ----- ----- ----- Informacoes importantes ----- ----- ----- ----- %



% ----- ----- ----- ----- Pacotes basicos ----- ----- ----- ----- %
\documentclass[a4paper,titlepage]{tcc_dcomp}
\usepackage[brazil,english]{babel}
\usepackage{procc,epsfig}
\usepackage{times}
\usepackage{verbatim}
\usepackage{mathtools}
% Para formulas matematicas
\usepackage{amssymb}
\usepackage{amsmath,amsfonts,amsthm}
% ----- ----- ----- ----- Pacotes basicos ----- ----- ----- ----- %



% ----- ----- ----- ----- Comandos para abreviacoes latinas ----- ----- ----- ----- %
\newcommand{\etal}{et~al.}
\newcommand{\ie}{i.~e.}
\newcommand{\eg}{e.~g.}
% ----- ----- ----- ----- Comandos para abreviacoes latinas ----- ----- ----- ----- %



% ----- ----- ----- ----- Pacotes de acentuacao em sistemas ISO8859-1 ----- ----- ----- ----- %
% Se estiver usando o Microsoft Windows ou linux com essa codificacao, 
% descomente essa linhas abaixo e comente as linhas referentes ao UTF8
%\usepackage[latin1]{inputenc} 
% Usar acentuacao em sistemas ISO8859-1, comentar a linha com  \usepackage[utf8x]{inputenc}
% ----- ----- ----- ----- Pacotes de acentuacao em sistemas ISO8859-1 ----- ----- ----- ----- %



% ----- ----- ----- ----- Pacotes de acentuacao em sistemas UTF8 ----- ----- ----- ----- %
% Para a maior parte das distribuicoes linux, usar a opcao utf8x 
% Lembrar de comentar as linha referente a ISO8859-1 acima!
\usepackage{ucs}
\usepackage[utf8x]{inputenc}
%\usepackage[utf8]{inputenc}
\usepackage[T1]{fontenc}
% ----- ----- ----- ----- Pacotes de acentuacao em sistemas UTF8 ----- ----- ----- ----- %



% ----- ----- ----- ----- Pacotes para tabelas e figuras ----- ----- ----- ----- %
\usepackage{multirow}
\usepackage{float}
\usepackage{graphicx,url}
\usepackage{subfig}
\usepackage{tabulary}
% Para tabelas longas (que ultrapassam uma pagina)
\usepackage{longtable}
\graphicspath{{Figuras/}}
%\input{psfig.sty}
% ----- ----- ----- ----- Pacotes para tabelas e figuras ----- ----- ----- ----- %



% ----- ----- ----- ----- Pacotes para codigo fonte ----- ----- ----- ----- %
\usepackage{listings}
\lstset{
	numbers          = left,
	stepnumber       = 1
	firstnumber      = 1,
	%numberstyle      = \tiny,
	extendedchars    = true,
	breaklines       = true,
	frame            = tb,
	basicstyle       = \footnotesize,
	stringstyle      = \ttfamily,
	showstringspaces = false
}
\renewcommand{\lstlistingname}{C\'odigo Fonte}
\renewcommand{\lstlistlistingname}{Lista de C\'odigos Fonte}
% ----- ----- ----- ----- Pacotes para codigo fonte ----- ----- ----- ----- %



%\selectlanguage{portuges}
\sloppy


%%%%%%%%%%%%%%%%%%%%%%%%%%%%%%%%%%%%%%%%%%%%%%%%%%%%%%%%%%%%%%%%%%%%%%%%%%%%%%%%


\begin{document}


% ----- ----- ----- ----- Informacoes para pagina de rosto ----- ----- ----- ----- %
% Titulo do trabalho em caixa alta
\Titulo{INSERIR TITULO EM MAIUSCULO}
% Nome do formando em caixa alta
\Autor{INSERIR NOME COMPLETO EM MAIUSCULO}
% Data da defesa por extenso
\Data{\today}
% Ano da defesa
\Ano{2016}
% Nome do(a) orientador(a)
\Orientador{Prof. Dr. (Me.) Nome do professor || Profa. Dra. (Ma.) Nome da professora}
% Universidade do(a) orientador(a)
\UniverOrientador{Universidade Federal de Sergipe (UFS)}

%Se possuir um professor co-orientador descomente essas duas linhas
%\CoOrientador{Prof. Dr. Nome Co-orientador}
%\UniverCoOrientador{Universidade Fictícia de Orientadores Brasileira (UFOB)}

%Nao utilizado
%\priAvaliador{Prof. Dr. Nome do primeiro Avaliador}
%\UniverpriAvaliador{Universidade Federal de Sergipe (UFS)}
%\segAvaliador{Prof. Dr. Nome do segundo Avaliador}
%\UniversegAvaliador{Universidade Federal de Sergipe (UFS)}
% ----- ----- ----- ----- Informacoes para pagina de rosto ----- ----- ----- ----- %



% ----- ----- ----- ----- Pagina de rosto ----- ----- ----- ----- %
\newpage
\cleardoublepage

\PaginadeRosto

\newpage
\cleardoublepage
% ----- ----- ----- ----- Pagina de rosto ----- ----- ----- ----- %



% ----- ----- ----- ----- Dedicatoria ----- ----- ----- ----- %
%[~debug] Verificar se existe esta parte do documento
%\begin{dedicatoria}
%Escreva aqui para quem você dedica este trabalho (interminável)\ldots
%\end{dedicatoria}

\newpage
\cleardoublepage
% ----- ----- ----- ----- Dedicatoria ----- ----- ----- ----- %



% ----- ----- ----- ----- Agradecimentos ----- ----- ----- ----- %
%[~debug] Verificar se existe esta parte do documento
%\begin{agradecimentos}
%\input{Documentos/Agradecimentos}
%\end{agradecimentos}

\newpage
\cleardoublepage
% ----- ----- ----- ----- Agradecimentos ----- ----- ----- ----- %



% ----- ----- ----- ----- Resumo ----- ----- ----- ----- %
\begin{resumo} 
Este projeto tem como objetivo facilitar a vida dos futuros formandos do departamento de computação da Universidade Federal de Sergipe.
Nos próximos capítulos você deverá aprender os conceitos básicos do LateX e como usar suas principais ferramentas.
Esperamos que seja proveitoso, e que você não mais precise usar o Word ou o LibreOffice Writer!! :-)
É aconselhável que, no resumo, seja escrita uma frase por linha!


\end{resumo}

\newpage
\cleardoublepage
% ----- ----- ----- ----- Resumo ----- ----- ----- ----- %



% ----- ----- ----- ----- Abstract ----- ----- ----- ----- %
\begin{abstract}
This project was made in order to make your life easier. Stop reading this and go learn something useful on the next chapters. Enjoy it! :-)
\end{abstract}

\newpage
\cleardoublepage
% ----- ----- ----- ----- Abstract ----- ----- ----- ----- %



% ----- ----- ----- ----- Listas de Figuras, Tabelas, Siglas, Codigos-fonte e Sumario ----- ----- ----- ----- %
\selectlanguage{brazil}

% Lista de Figuras
\listoffigures
% Lista de Tabelas
\listoftables
% Lista de Siglas
\ListadeSiglas
% Lista de Codigos-fonte
%\lstlistoflistings
\renewcommand{\contentsname}{Sum\'ario}
\Sumario
% ----- ----- ----- ----- Listas de Figuras, Tabelas, Siglas, Codigos-fonte e Sumario ----- ----- ----- ----- %



% ----- ----- ----- ----- Introducao ----- ----- ----- ----- %
\Introducao

\newpage
\cleardoublepage
% ----- ----- ----- ----- Introducao ----- ----- ----- ----- %



% ----- ----- ----- ----- Estilo do Cabecalho e Hifenizacao ----- ----- ----- ----- %
% Definicoes do cabecalho:
% Secao do lado esquerdo
% Numero da pagina do lado direito
\pagestyle{fancy}
\addtolength{\headwidth}{\marginparsep}\addtolength{\headwidth}{\marginparwidth}\headwidth = \textwidth
\renewcommand{\chaptermark}[1]{\markboth{#1}{}}
\renewcommand{\sectionmark}[1]{\markright{\thesection\ #1}}\lhead[\fancyplain{}{\bfseries\thepage}]%
	     {\fancyplain{}{\emph{\rightmark}}}\rhead[\fancyplain{}{\bfseries\leftmark}]%
             {\fancyplain{}{\bfseries\thepage}}\cfoot{}

% Hifenizacao - Colocar lista de palavras que nao devem ser separadas 
% e que nao estao no dicionario portugues.
% Palavras do dicionario portugues sao separadas corretamente pelo lateX
\hyphenation{ Hardware Software }
% ----- ----- ----- ----- Estilo do Cabecalho e Hifenizacao ----- ----- ----- ----- %



% ----- ----- ----- ----- Capitulos ----- ----- ----- ----- %
\chapter{Conhecendo o LateX}\label{capConhecendoLatex}
Neste capítulo você irá conhecer coisas de outro mundo. Você irá escrever um livro como se estivesse programando. Eu sei! Você deve estar pensando: "Que loco cara!" mas é isso mesmo!!!

Nada mais de ter que se preocupar com formatação, como organizar capítulos, referências cruzadas, entre outras maravilhas que tiram o socego de quem detesta escrever artigos. Mas nem tudo é alegria\ldots Neste documento você também irá sentir saudade do corretor ortográfico do Word com aquele toque de correção semantica\ldots Bons tempos quando você tinha alguém que te dizia que você estava escrevendo errado. Só pra citar~\cite{camargo}.

Por isso lembre-se! Se você escrever algo do tipo: "\underline{Chamei a secretaria para me ajudar, ela é uma pessoa muito colaborativa}". O LateX não irá te avisar que existe um erro semântico alí. E dependendo do editor que você estiver usando não irá nem indicar um ero ortográfico. Por isso fique atento(a)!

Ah e antes que me esqueça... Não podemos esquecer de uma tradição tão antiga:

\textbf{Olá mundo!}

É importante que conforme você avance na leitura deste documento, também fique atento no que está ocorrendo no \textbf{\textit{código fonte}} do mesmo. Pois iremos apresentar as novas funcionalidades a partir de exeplos.


%\chapter{Introdução}\label{introducao}

\section{Motivação}\label{introducao:motivacao}

\subsection{Implementações existentes}\label{introducao:motivacao:implementacoes}

\section{Escopo}
%\input{Documentos/Conceitos}
%\input{Documentos/Metodologia}
%\input{Documentos/Resultados}
%\input{Documentos/Conclusao}
% ----- ----- ----- ----- Capitulos ----- ----- ----- ----- %



% ----- ----- ----- ----- Bibliografia ----- ----- ----- ----- %
% Troca do termo Bibliografia para Referencia.
\renewcommand{\bibname}{REFER\^ENCIAS}
% Adicao da pagina da Referencia no Sumario
\addcontentsline{toc}{chapter}{Refer\^encias}
% Estilo de bibliografia: plain, unsrt, alpha, abbrv, etc.
% Para mais, visite: https://en.wikibooks.org/wiki/LaTeX/Bibliography_Management#Bibliography_styles
\bibliographystyle{plain}
% Coloca todas as referencias
%\nocite{*} 
% Arquivos com as entradas bib.
\bibliography{Documentos/Referencias}
% ----- ----- ----- ----- Bibliografia ----- ----- ----- ----- %



% ----- ----- ----- ----- Apendice ----- ----- ----- ----- %
% Caso haja algum apendice, descomente a linha abaixo e crie o arquivo apendices.tex.
\appendix
%\input{Documentos/Apendices.tex}
% ----- ----- ----- ----- Apendice ----- ----- ----- ----- %

\end{document}